\documentclass[11pt,a4paper]{article}

\usepackage[utf8]{inputenc}
\usepackage[margin=1.2cm]{geometry}
\usepackage{setspace}
\usepackage{xcolor}
\usepackage{hyperref}
\usepackage{enumitem}
\usepackage[default]{raleway}
\usepackage{fontawesome5}
\usepackage{fancyhdr}
\usepackage{tikz}
\usepackage{lastpage}

% Color definitions
\definecolor{darkblue}{RGB}{31, 78, 121}
\definecolor{accentgreen}{RGB}{76, 175, 80}
\definecolor{lightgray}{RGB}{245, 245, 245}
\definecolor{mediumgray}{RGB}{200, 200, 200}

% PDF metadata
\hypersetup{colorlinks=true, linkcolor=darkblue, urlcolor=darkblue}

% Remove page numbers
\pagestyle{fancy}
\fancyhf{}
\fancyfoot[C]{\small (\thepage/\pageref*{LastPage})}
\renewcommand{\headrulewidth}{0pt}
\setlength{\footskip}{20pt}

% Custom commands with improved design
\newcommand{\sectiontitle}[1]{%
    \vspace{6pt}
    {\Large\bfseries\color{darkblue}#1}
    \nopagebreak
    \vspace{3pt}
    {\color{accentgreen}\hrule height 2pt}
    \vspace{4pt}
}

\newcommand{\jobtitle}[5]{%
    \textbf{#1} \hfill \textit{\small #2}\\
    {\color{darkblue}\textit{#3}} \hfill {\small #4}\\
    \textit{#5}\\[8pt]
}

\newcommand{\degreeline}[4]{%
    \textbf{#1} \hfill \textit{\small #2}\\
    {\small #3} \hfill {\color{darkblue}\textbf{#4}}\\[3pt]
}

\setlist[itemize]{leftmargin=0.5cm, itemsep=1pt, parsep=0pt}

\begin{document}

\raggedright

% Header with improved spacing
\begin{minipage}{0.7\textwidth}
    {\fontsize{32}{36}\selectfont\bfseries\color{darkblue}Marc Schneider}
    
    {\large Wissenschaftlicher Mitarbeiter | Luft- und Raumfahrttechnik}
    \\
    % {\normalsize\color{accentgreen}Guided Missiles, Guidance Laws \& Navigation Systems}
    {\normalsize\color{accentgreen}Lenkung, Navigation \& Regelung | Maschinelles Lernen \& Sensorfusion}
    
    \vspace{4pt}
    
    {\small
    \faPhone\ +49 160 2421697 \quad \faEnvelope\ marc@marc-schneider.de \quad \faMapMarker\ Stuttgart, Deutschland
    
    \faLinkedinIn\ \href{https://www.linkedin.com/in/marc-schneider-96a8a3238/}{LinkedIn} \quad \faResearchgate\ \href{https://www.researchgate.net/profile/Marc-Schneider-23}{ResearchGate}
    }
\end{minipage}
\hfill
\begin{minipage}{0.2\textwidth}
    \includegraphics[width=\linewidth]{Profilbild.jpg}
\end{minipage}

% \vspace{6pt}

% {\color{black}\hrule height 2pt}

\vspace{2pt}

% Professional Summary
\sectiontitle{Berufliches Profil}
Luft- und Raumfahrtingenieur mit bewiesener Expertise in Industrie- und Forschungsprojekten von der Planung bis zur Fertigstellung. Erfahrener Dozent mit Erfahrung in der Durchführung von Seminaren und Betreuung von Studierenden aller Ebenen. Spezialisiert auf die Verbindung von klassischer Lenkung, Navigation und Regelung (GNC) mit Maschinellem Lernen (ML), mit Fokus auf Trajektorienvorhersage, Lenksysteme für Flugkörper, Sensorfusion und autonome Systeme. Starker Hintergrund in mathematischer Modellierung, Simulation und Algorithmenentwicklung.

\vspace{2pt}

% Work Experience
\sectiontitle{Berufliche Erfahrung}

\jobtitle{Wissenschaftlicher Mitarbeiter / Promovend}{Aug. 2020 -- Juli 2026}{Institut für Fluggechanismus und Regelung, Universität Stuttgart}{Stuttgart, Deutschland}{Forschungsschwerpunkt: Lenkung, Navigation \& Regelung und Maschinelles Lernen für Luft- und Raumfahrtanwendungen.}
% \hspace{1em}\textbf{Research Focus:} 

% \vspace{-10pt}

\vspace{2pt}


\hspace{-0.0em}\textbf{Projekte:}
% \vspace{-9pt}

\begin{itemize}[leftmargin=1.0cm]
    \item \textbf{Maschinelles Lernen für kooperative Lenkflugkörper} -- Stochastische Trajektorienvorhersage mit Normalizing Flows (\textit{Wo wird sich ein Ziel in der Zukunft befinden?}), Lösung von dynamischen Waffenzielzuordnungsproblemen (\textit{Wie lenkt man mehrere kooperierende Lenkflugkörper auf mehrere Ziele?})

    \item \textbf{Sensorgestützte Landung auf Asteroiden mit Reinforcement Learning} -- Nutzung von Lidar-Messungen in einem LSTM-RL-Agent zur Ersetzung der nominalen GNC-Algorithmen im Fehlerfall (\textit{Wie landet man ein Raumfahrzeug auf einem Asteroiden, indem man nur verrauschte Sensordaten nutzt?})

    \item \textbf{Sensorfusion für Lenkflugkörper-Navigation} -- Multi-Sensor-Integration mit IMU, GNSS, Radar und IR unter dem Einfluss verzögerter Messungen (\textit{Wie fusioniert man Sensorinformationen, um den Zustand des Lenkflugkörpers und des Ziels besser abzuschätzen?})

    \item \textbf{Multi-Hypothesen-Lenkung für Flugkörper} -- Berechnung von Lenkbefehlen unter Berücksichtigung mehrerer Hypothesen über das Zielverhalten (\textit{Wie lenkt man einen Flugkörper bei unsicheren Zielmanövern?})
\end{itemize}

\hspace{0.0em}\textbf{Lehrtätigkeit \& Kursentwicklung:}
\begin{itemize}[leftmargin=1.0cm]
    \item \textbf{Schätzverfahren-Seminar} -- Kursleiter und Inhaltsverantwortlicher (2021 -- 2023)
    \item \textbf{Aerobotics Seminar} -- Dozent und Lehrplanentwickler (2021--2024), Betreuung von Reinforcement Learning und Model Predictive Control Gruppen
\end{itemize}

\vspace{2pt}

\jobtitle{Wissenschaftliche Hilfskraft}{Sep. 2018 -- Nov. 2018}{Institut für Fluggechanismus und Regelung, Universität Stuttgart}{Stuttgart, Deutschland}{Weiterentwicklung des DA40-Flugsimulators und Implementierung eines Hubschraubersimulationsmodells.}

% \vspace{-9pt}

\jobtitle{Wissenschaftliche Hilfskraft}{Mai 2018 -- Aug. 2018}{Institut für Fluggechanismus und Regelung, Universität Stuttgart}{Stuttgart, Deutschland}{Tutor für Vorlesung „Flugregelungsentwurf": Verbesserung des Flugsimulators und Unterstützung bei Übungen.}


% \vspace{-9pt}

\jobtitle{Praktikant}{Apr. 2017 -- Aug. 2017}{MAHLE GmbH}{Stuttgart, Deutschland}{Auslegung von Wärmemanagementsystemen für Lithium-Ionen-Batterien. Analyse und Optimierung von Kühlsystemen.}

% \vspace{-9pt}

% \jobtitle{Intern}{Jun. 2014 -- Aug. 2014}{ZENTNER Elektrik-Mechanik GmbH}{Freiburg, Germany}{Introductory internship: Metal processing, machining, and basic mechanical engineering principles.}

% \vspace{5pt}

\newpage

% Education
\sectiontitle{Ausbildung}

\degreeline{Masterarbeit}{2019 -- 2020}{Path Planning and Obstacle Avoidance for Multiple UAVs Using POMDP, \textit{Note: 1,0} \\ }{\qquad \qquad \qquad \qquad \qquad \qquad \quad \qquad \qquad \qquad \qquad \mbox{Queensland University of Technology, Brisbane, Australien}}

\vspace{5pt}

\degreeline{Master of Science in Luft- und Raumfahrttechnik}{2018 -- 2020}{Gesamtnote: 1,3 (Top 13\%) \\Spezialisierungen: Experimentelle und numerische Simulationsmethoden in der LRT, Flugführung und Systemtechnik in der LRT\phantom{very long thesis description text to balance the alignment}}{Universität Stuttgart}

\vspace{5pt}

\degreeline{Bachelor of Science in Luft- und Raumfahrttechnik}{2014 -- 2018}{Gesamtnote: 1,7 (Top 4\%) \\Arbeit: Entwurf einer Simulationsumgebung zur kooperativen Aufwindschätzung, \textit{Note: 1,0}}{Universität Stuttgart}

\vspace{5pt}

\degreeline{Abitur}{2006 -- 2014}{Gesamtnote: 1,0}{Kepler Gymnasium Freiburg}

\vspace{15pt}


% Skills
\sectiontitle{Technische Fähigkeiten \& Kernkompetenzen}


\textbf{Forschung \& Entwicklung:}  
Lenkung, Navigation \& Regelung (GNC), Flugmechanik, Lenksysteme, Kalman-Filter, Sensorfusion, Regelungstheorie, Schätztheorie, Multi-Agent-Systeme, Reinforcement Learning, Maschinelles Lernen, Algorithmen, Simulationen

\vspace{0.5em}



\vspace{6pt}

\textbf{Programmiersprachen:}
\begin{itemize}
    \item \textbf{Experte:} Python (PyTorch, NumPy, SciPy, Scikit-learn, Matplotlib), MATLAB, Simulink, \LaTeX
    \item \textbf{Fortgeschrittene Kenntnisse:} Git, Bash
    \item \textbf{Mittlere Kenntnisse:} C/C++
\end{itemize}

\vspace{6pt}

\textbf{Sprachen:}
\begin{itemize}
    \item Deutsch (C2 -- Muttersprache)
    \item Englisch (C1 -- Verhandlungssicher)
    \item Französisch (B1 -- Mittelstufe)
\end{itemize}

\vspace{6pt}

\textbf{Tools \& Frameworks:}  
Versionskontrollsysteme (Git), Cloud Computing, Deep Learning Frameworks (PyTorch), Experiment Tracking (Weights \& Biases), Hyperparameter-Optimierung (Optuna), GitHub Copilot


\vspace{15pt}

% Supervised Theses
\sectiontitle{Betreute Abschlussarbeiten}

\begin{enumerate}[label=\textbf{[\arabic*]}, leftmargin=0.8cm, itemsep=3pt]
    \item (2025) \textit{Stochastic Trajectory Prediction of Civil Aircraft using Conditional Normalizing Flows} -- Masterarbeit
    \item (2025) \textit{Development and Implementation of a GPS Monitoring Algorithm} -- Masterarbeit
    \item (2024) \textit{Reachability Prediction Using Neural Networks} -- Masterarbeit
    \item (2023) \textit{Autonome Ausweichmanöver gegen Lenkflugkörper durch Bestärkendes Lernen} -- Masterarbeit
    \item (2022) \textit{Sensor Fusion for Distributed Directional Measurements} -- Masterarbeit
    \item (2022) \textit{Prädiktives Lenkverfahren unter Berücksichtigung des induktiven Luftwiderstands} -- Bachelorarbeit
    \item (2021) \textit{Model Predictive Control Guidance Law for Guided Missiles} -- Masterarbeit
\end{enumerate}

\vspace{15pt}



% new page
\newpage
% Publications
\sectiontitle{Veröffentlichungen}

\begin{enumerate}[label=\textbf{[\arabic*]}, leftmargin=0.8cm, itemsep=3pt]

    \item (2026). "Reachability Prediction of Guided Missiles Using Active Learning of Artificial Neural Networks." \href{https://arc.aiaa.org/doi/abs/10.2514/6.2026-2166}{AIAA}
    \item (2025). "Many-vs-Many Missile Guidance via Virtual Targets" (preprint). \href{https://arxiv.org/abs/2511.02526}{arXiv}
    
    \item (2025). "Autonomous Evasive Maneuvers Against Missiles Through Reinforcement Learning." \href{https://www.researchgate.net/publication/387995943_Autonomous_Evasive_Maneuvers_against_Missiles_through_Reinforcement_Learning}{ResearchGate}
    
    \item (2025). "Real-Time Cooperative Target Allocation for Guided Missiles: Leveraging Optimal Control and Multiple Trajectories." \href{https://arc.aiaa.org/doi/10.2514/6.2025-1907}{AIAA}
    
    \item (2025). "Virtual Target Trajectory Prediction for Stochastic Targets." \href{https://arxiv.org/pdf/2504.01851}{arXiv}
    
    \item (2024). "Trajectory Prediction for Missile Targets: A Probabilistic Approach Using Machine Learning." \href{https://www.researchgate.net/publication/381520100_Trajectory_Prediction_for_Missile_Targets_A_Probabilistic_Approach_Using_Machine_Learning}{ResearchGate}
    
    \item (2023). "Radar-Aided Inertial Navigation with Delayed Measurements." \href{https://www.researchgate.net/publication/360484762_Radar-Aided_Inertial_Navigation_with_Delayed_Measurements}{ResearchGate}
    
    \item (2023). "Enhancing Target Acquisition in Long-Range Missiles through Multi-Sensor Fusion." \href{https://www.researchgate.net/publication/377395801_Enhancing_Target_Acquisition_in_Long-Range_Missiles_through_Multi-Sensor_Fusion}{ResearchGate}
    
    \item (2022). "Multi-Hypothesis Guidance With Interacting Multiple Model Filter." \href{https://www.researchgate.net/publication/357595202_Multi-Hypothesis_Guidance_With_Interacting_Multiple_Model_Filter}{ResearchGate}
\end{enumerate}

\vspace{15pt}

% \sectiontitle{Personal}

% \begin{itemize}
%     \item Developed and host \href{https://spy-signal.com}{spy-signal.com}, an information website and newsletter focused on leveraged ETFs.
%     \item Strong interest in local large language models (LLMs) and their applications in research and development.
% \end{itemize}

\vspace{15pt}

\vfill

\begin{center}
    \small
    \textit{Stuttgart, 3. Januar 2026}
\end{center}

\end{document}
